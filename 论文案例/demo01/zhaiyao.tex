\begin{abstract}         %摘要部分
    \small\centering{俗话说,苦尽甘来,乐极生悲,其实痛苦与快乐,就象一对生死怨家,总是势不两立的;又象是一对孪生兄弟,如此亲密。所以痛苦过后,随之而来的,就是幸福、快乐,就好比不经历风雨,怎么能见彩虹,不经一番严寒彻骨,又怎能得到梅花的扑鼻香味呢?所谓苦尽甘来,也就有了越王勾践卧薪尝胆之后的国强民盛。而快乐过后,痛苦也随之而来,所谓乐极生悲,于是也就有了范进中举后的发疯。
    有句俗语说,三十年河东,三十年河西;还有句俗语说,风水轮流转,明年到我家;(励志歌曲)由此可见,世间的一切,并没有一个定数,这包括了痛苦与快乐。所以,也就没有必要,因为短暂的快乐而喜形于色。得意忘形,甚至沉迷于内,更没必要为一时的痛苦而垂头丧气,由此而变得意志浮沉。
    
    世人不明,总是不断地去追求快乐,追求幸福,却不愿意面对痛苦。没钱的希望发财,有钱的希望更多,有名的希望更响,当官的希望官做的更大,而一旦当这些失去时,巨大的失落,就成了一种剜心的痛苦。而过度的追求快乐时,那已经不是一种快乐,而是一种不断膨胀的欲望,当这种欲望冲昏了头脑并占据了自己的思想之时,最终会被这种欲望埋葬了自己。
    }
\end{abstract}